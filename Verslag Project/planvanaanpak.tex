\chapter{Plan van aanpak}
\label{ch:pva}

Op dinsdag 4 juli kwam het team fysiek samen om de opzet van de oplossing te bespreken. Bij deze meeting werd gebrainstormd welke richting we willen uitgaan en hoe dit zal worden geïmplementeerd. Er werd begonnen met het bekijken van de online video door \textcite{eva_video} en notities te nemen bij belangrijke punten die aangehaald werden door de vertegenwoordiger van Eva. Hierdoor kwam elk teamlid aan een eigen lijst en werd op basis daarvan iedereen zijn input besproken en vergeleken. 

Op basis van een gezamelijk verkregen lijst werd begonnen aan een groot aantal user stories en use cases die in de app zouden kunnen komen. Een groots idee dat daaruit naar voor kwam was een soort spel te implementeren waarbij mensen per regio kunnen proberen om de regio te zijn met de hoogste veganistische score van België. Na verder te beraadslagen hieromtrent kwamen we tot de conclusie dat dit gezien de tijdspanne, en ook het feit dat niemand binnen het team de design skills had om zoiets te realiseren, er toch een andere aanpak zal moeten volgen.

Er werd altijd in het achterhoofd gehouden dat dit ook een applicatie moet worden die aanspreekbaar is voor een heel breed publiek. Uiteindelijk kwamen we uit op drie grote delen. De applicatie zal bestaan uit de uitdagingen, waarbij de vooruitgang van de gebruiker zal gevolgd worden. Elke vijfde uitdaging zal ook een beloning bevatten die de gebruiker kan innen als hij de opdracht tot een succesvol einde weet te brengen. Het tweede hoofddeel bestaat uit een lijst met restaurants zodat iemand die geen zin heeft om deel te nemen aan de opdrachten toch goede locaties kan vinden om veganistisch te gaan eten. 

Het derde deel, en ons paradepaardje, genaamd Vegagram is een soort van sociale media geïmplementeerd in de applicatie. Gebruikers zullen de mogelijkheid hebben tot het nemen van afbeeldingen en deze online te zetten. Dan hebben ze de keuze of ze deze publiek beschikbaar wensen te maken, zodat andere gebruikers de afbeeldingen kunnen zien. Op de publiek beschikbare posts kan men ook aantonen wat ze van de afbeelding vinden door middel van een "vind ik leuk" knop.

Bij al dit hoort ook een beheerderskant deze staat online en is beschikbaar via volgende link \autocite{dashboard}. Dit staat toe om binnen EVA vzw alles omtrent de app te beheren. Dit omvat aanpassen, updaten, verwijderen en toevoegen van alles binnen de app. Hiervoor wouden we gaan voor een dashboard look and feel. Die intuïtief en eenvoudig te gebruiken is. Zo kan een administrator inloggen op het dashboard om daar, bijvoorbeeld, de lijst van restaurants aan te passen of een nieuwe toe te voegen. Voor een duidelijk overzicht te garanderen tonen we kleine gepagineerde lijsten met filter opties en duidelijke knoppen die toelaten de gewenste actie uit te voeren. De best practices en schema's met betrekking tot de beheerderskant zijn terug te vinden in de technische beschrijving.
