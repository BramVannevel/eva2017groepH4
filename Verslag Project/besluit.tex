\chapter{Conclusie}
\label{ch:conclusie}

Voor de tweede zit koos het team om met dezelfde teamleden te blijven als bij eerste zit. Zo hebben we de mogelijkheid gehad tot het aanpassen en implementeren van betere opties, die voorgesteld zijn door de docenten doorheen de eerste zit, binnen onze groep. Zo werd het Agile principe, en werken beter gehanteerd en gerespecteerd. Dit bekwamen we door de geleerde principes in de praktijk om te zetten. Zoals bijvoorbeeld het gebruik van een scrummaster en het, veel, beter gebruik van Trello en de burndowns.

Ook binnen Android werd van aan de start beter rekening gehouden met de best practices en guidelines. Er werd van in het begin gewerkt met een overzichtelijke structuur binnen de code. Ook werd veel gebruik gemaakt van bekende design patterns. Door hieraan te voldoen is de code onderhevig aan makkelijke uitbreidbaarheid en onderhoud. Het is ook eenvoudig voor andere teamleden om deel te nemen aan het Android gedeelte en verder te werken aan vereisten zonder lang te moeten stilstaan bij het ontwerp van de code. Een mooi voorbeeld hiervan is de Retrofit builder. Alles is mooi afgeschermd en opgesplitst zodat het eenvoudig te implementeren is in nieuwe Activities of Fragments. Ook werd beter rekening gehouden met de levenscyclus van Android door attributen beter te beheren af te sluiten binnen de onDestroy van een Activity. Bijvoorbeeld de CompositeDisposables en Butterknife methodes.

Met betrekking tot de MEAN stack werden dezelfde principes gehanteerd als in de eerste zit. Toen was de beheerderskant al codegewijs goed. Hier hebben we dan vooral ingespeeld op een betere User Interface en nog enkele best practices aan toegevoegd. Zoals de afbeeldingen die opgeslaan worden op de server en via verwijzing in de databank terecht komen. Ook werden voor dit project User Roles toegevoegd aan de applicatie. Zo worden personen die hun registreren via de app standaard een algemene gebruiker. Deze hebben dan geen toegang tot de beheerderskant.